\documentclass{beamer}
\usetheme{Madrid}

\title{EPL448 Data Mining Project: Predicting Prices for Used Cars}
\author{Andreas Hadjoullis\\Julios Fotiou}
\institute{Department of Computer Science \\ University of Cyprus}
\date{\today}

\begin{document}

\begin{frame}
    \titlepage
\end{frame}

\begin{frame}{Outline}
    \tableofcontents
\end{frame}

\section{Introduction}
\begin{frame}{Introduction}
    \begin{itemize}
        \item The used car market is huge and rapidly changing. Pricing a used car correctly is crucial for both sellers and buyers.
        \item \textbf{Dataset:} Our dataset initially contained 371,528 rows and 21 columns, sourced from Kaggle: 
        \item Sourced from \href{https://www.kaggle.com/datasets/thedevastator/uncovering-factors-that-affect-used-car-prices/data}{\textbf{Kaggle}}.
        \item \textbf{Problem:} Accurately predict the selling price of a used car based on its features (brand, model, year, gearbox, etc.).
        \item \textbf{Objectives:}
            \begin{itemize}
                \item Explore and analyze the dataset.
                \item Engineer features and preprocess data.
                \item Build and evaluate predictive models.
            \end{itemize}
        \item \textbf{Challenges:} Presence of outliers, irrelevant features, and highly skewed price distribution.
    \end{itemize}
\end{frame}

% Initial Data Exploration
\section{Initial Data Exploration}
\begin{frame}{Initial Data Exploration}
    \begin{itemize}
        \item Kept only rows with realistic prices (1,000–200,000) for effective modeling.
        \item \texttt{offerType} and \texttt{seller} each had two categories, but one category dominated almost entirely (the minority category had only a handful of rows).
        \item \texttt{nrOfPictures} had only a single unique value for all entries.
        \item These columns were removed as they provided no useful information for prediction.
    \end{itemize}
\end{frame}

\begin{frame}{Handling Missing Models Values}
    \begin{itemize}
        \item The \texttt{model} column contained many missing values, making it unreliable for use directly.
        \item The \texttt{name} column, though messy and user-entered, always existed and typically included the brand, model, and extra info.
        \item \textbf{Solution:} We extracted the car model from the \texttt{name} field using the following steps:
        \begin{itemize}
            \item Collected a comprehensive list of car brands and their possible models from the web.
            \item Cleaned and tokenized the name, then generated all 1–3 word n-grams.
           \item Matched these n-grams against known models for that brand using fuzzy matching.
            \item If a strong match was found (above a similarity threshold), we filled in the missing \texttt{model}.
        \end{itemize}
    \end{itemize}
\end{frame}

\begin{frame}{Results of Model Extraction}
    \begin{itemize}
        \item \textbf{Before extraction:} 12,148 missing values in the \texttt{model} column.
        \item Using n-gram and fuzzy matching based on the car brand and name, we reduced this to 4,114 missing values.
        \item \textbf{Limitations:}
        \begin{itemize}
            \item 2,857 entries could not be filled because the brand was \texttt{sonstige\_autos} (“other cars”).
            \item 1,257 entries had insufficient information in the \texttt{name} field.
        \end{itemize}
        \item \textbf{Final step:} To ensure data quality, we dropped all rows where the \texttt{model} value remained missing.
        \item This resulted in a clean and consistent dataset for further analysis and modeling.
    \end{itemize}
\end{frame}

\begin{frame}{Target Variable Analysis}
    \begin{itemize}
        \item The primary target is the car's selling \texttt{price}.
        \item \textbf{Distribution:} Price distribution is highly right-skewed, with many extreme outliers.
        \item \textbf{Normality tests:} All tests (Anderson-Darling, Kolmogorov-Smirnov, D’Agostino-Pearson, Jarque-Bera, Lilliefors) indicate that \texttt{price} does \textbf{not} follow a normal distribution.
        \item \textbf{Skewness:} 5.14 (very high, confirms extreme right skew).
        \item \textbf{Implications:}
            \begin{itemize}
                \item Model will perform best on low/average prices, but may struggle with high-price cars due to data imbalance.
                \item Considered unskewing techniques such as Box-Cox and Yeo-Johnson.
            \end{itemize}
    \end{itemize}
\end{frame}


\begin{frame}{Categorical Features Analysis}
    \begin{itemize}
        \item Explored distributions of key categorical variables using bar plots and boxplots.
        \item \textbf{ABtest (\texttt{abtest}):} Both categories ("test" and "control") are similarly distributed, but show no influence on price.
        \item \textbf{Vehicle Type (\texttt{vehicleType}):} Distribution is dominated by a few types (kleinwagen, limousine, kombi), which strongly influence price.
        \item \textbf{Gearbox (\texttt{gearbox}):} Majority are manual, but automatic is also significant. Gearbox type shows high correlation with price.
        \item \textbf{Month of Registration (\texttt{monthOfRegistration}):} Mostly balanced, except for "0" (likely missing values).
    \end{itemize}
\end{frame}

\begin{frame}{Categorical Features Analysis}
    \begin{itemize}
        \item \textbf{Fuel Type (\texttt{fuelType}):} Benzin dominates, diesel also frequent. Rare types (lpg, andere, cng) can be grouped as "other", but "hybrid" and "elektro" show distinct price patterns.
        \item \textbf{Not Repaired Damage (\texttt{notRepairedDamage}):} "Nein" dominates, but "ja" shows much lower prices, indicating a strong correlation with the target.
        \item \textbf{Model \& Brand:} Both have high correlation with price and large within-category deviations, especially for brands with many models (e.g., Porsche 911 vs Cayenne).
        \item \textbf{General observations:}
            \begin{itemize}
                \item Many categorical features are unbalanced; stratified sampling may be needed for fair modeling.
                \item Outliers are present in every category.
                \item Grouping or discarding rare categories considered to improve model stability.
            \end{itemize}
    \end{itemize}
\end{frame}


\begin{frame}{Numerical Features Analysis for All Values}
    \begin{itemize}
        \item Initial analysis revealed extreme skewness and non-normality in several numerical features:
        \begin{itemize}
            \item \texttt{yearOfRegistration}: Skewness 106.79
            \item \texttt{powerPS}: Skewness 61.90 
            \item \texttt{kilometer}: Skewness -1.33
            \item \texttt{postalCode}: Skewness $\approx$ 0.02, no strong skew
        \end{itemize}
        \item \textbf{Normality tests:} All tests (Anderson-Darling, Kolmogorov-Smirnov, etc.) strongly rejected the hypothesis of normal distribution for all features.
    \end{itemize}
\end{frame}


\begin{frame}{Numerical Features Analysis for Valid Values}
    \begin{itemize}
        \item After our initial analysis, we carefully checked each feature for valid value ranges:
        \begin{itemize}
            \item For \texttt{yearOfRegistration}, we kept only values from 1920 to 2016, since \texttt{DateCreated} was never after 2016 and earlier years were invalid.
            \item For \texttt{powerPS}, we kept only values less than or equal to 1500, as it is not realistic for a car under 200,000€ to have more power.
            \item For \texttt{kilometer} and \texttt{postalCode}, we confirmed all values were within realistic ranges.
        \end{itemize}
        \item This cleaning significantly reduced skewness:
        \begin{itemize}
            \item \texttt{yearOfRegistration}: -1.65
            \item \texttt{powerPS}: 2.79
            \item \texttt{kilometer}: -1.27
            \item \texttt{postalCode}: $\approx$ 0
        \end{itemize}
    \end{itemize}
\end{frame}

\begin{frame}{Correlation Analysis}
    \begin{itemize}
        \item \textbf{kilometer} and \textbf{price} have a moderate negative correlation: cars with more kilometers tend to be less expensive.
        \item \textbf{powerPS} and \textbf{price} have a moderate positive correlation: higher power typically means a higher price.
        \item \textbf{yearOfRegistration} and \textbf{price} have a moderate positive correlation: newer cars are generally more expensive.
        \item \textbf{yearOfRegistration} is negatively correlated with \textbf{kilometer}: newer cars generally have lower mileage.
        \item \textbf{postalCode} shows little or no correlation with price, indicating it may not be a useful predictor.
        \item These correlations align with expectations and support feature selection for modeling.
    \end{itemize}
\end{frame}

% Splitting Data
\section{Data Splitting}
\begin{frame}{Data Cleaning}
    \begin{itemize}
        \item \textbf{Set invalid values to missing:}
        \begin{itemize}
            \item \texttt{yearOfRegistration}: Years $<$ 1920 and $>$ 2016 are not realistic for used cars in our dataset. Setting these to NaN ensures we only use plausible registration years for modeling.
            \item \texttt{powerPS}: Values below 4.5 or above 1500 are outside the range for real vehicles (especially considering our price limits). Marking these as missing, removes obvious errors and extreme outliers.
            \item \texttt{monthOfRegistration}: A value of 0 is not a valid month and likely indicates missing or unknown data, so we treat it as NaN.
        \end{itemize}
        \item \textbf{Grouped rare \texttt{fuelType} categories:}
        \begin{itemize}
            \item Categories like "lpg" and "cng" had very few entries, making it hard for the model to learn about them.
            \item We combined these into "andere" ("other") to improve statistical reliability.
            \item "hybrid" and "elektro" were kept separate despite being rare, as they showed distinct price patterns and might be important for predictions.
        \end{itemize}
    \end{itemize}
\end{frame}

\begin{frame}{Feature Engineering: Age Columns}
    \begin{itemize}
        \item Added two new features:
        \begin{itemize}
            \item \texttt{adLifespan}: Time (in days) between when an ad was first seen and last seen.
            \item \texttt{carAge}: Age of the car in days at the time the ad was created.
        \end{itemize}
        \item Replaced date columns with these new features for modeling convenience.
        \item Checked and corrected for negative or zero values in \texttt{carAge}.
        \item Dropped columns: \texttt{registrationDate}, \texttt{dateCrawled}, \texttt{dateCreated}, \texttt{lastSeen}.
        \item \texttt{adLifespan} remained slightly right-skewed (skewness: 0.96) and did not follow a normal distribution.
        \item \texttt{carAge} and \texttt{yearOfRegistration} had a perfect negative correlation ($r = -1.00$), so we dropped \texttt{yearOfRegistration}.
    \end{itemize}
\end{frame}

\begin{frame}{Train/Test Split and Handling Imbalance}
    \begin{itemize}
        \item Split the dataset into train and test sets (80/20 split).
        \item Rare categories in \texttt{fuelType} ("hybrid" and "elektro") were oversampled in the training set (20x) to address class imbalance and ensure the model learns from these cases.
        \item After oversampling, shuffled the training set to mix synthetic and real rows.
    \end{itemize}
\end{frame}

\begin{frame}{Scaling and Target Transformation}
    \begin{itemize}
        \item Applied robust scaling to numerical features: \texttt{carAge}, \texttt{powerPS}, \texttt{kilometer}, \texttt{adLifespan}.
        \item Checked post-scaling standard deviation to confirm effective normalization.
        \item Transformed the target variable (\texttt{price}) using Box-Cox, as all values were positive, to reduce skewness and improve model performance.
    \end{itemize}
\end{frame}

\section{Feature Selection}
\begin{frame}{Feature Selection and Dimensionality Reduction}
    \begin{itemize}
        \item \textbf{Sequential Forward Selection (SFS):}
        \begin{itemize}
            \item Used SFS with XGBoost to select the most predictive features.
            \item Best feature set: vehicleType, gearbox, powerPS, model, kilometer, fuelType, brand, notRepairedDamage, adLifespan, carAge.
            \item Features \texttt{postalCode} and \texttt{monthOfRegistration} were not selected, confirming their low predictive value.
        \end{itemize}
        \item \textbf{Extra Trees Classifier:}
        \begin{itemize}
            \item Attempted to compute feature importances.
            \item Kernel repeatedly crashed with larger \texttt{n\_estimators}, so results are unstable and not reliable for ranking.
        \end{itemize}
        \item \textbf{Dimensionality Reduction:}
        \begin{itemize}
            \item \textbf{PCA} not used: only suitable for continuous numerical data, but our dataset contains many categorical variables.
            \item \textbf{SVD} not used: most effective for large, sparse matrices, which does not apply here.
            \item Not a limitation, as we do not have too many features.
        \end{itemize}
    \end{itemize}
\end{frame}

\begin{frame}{Dataset Versions and Preprocessing Strategies}
    \begin{itemize}
        \item Created multiple dataset versions (V1–V8) to explore the impact of different preprocessing strategies:
        \begin{itemize}
            \item Imputation strategies: mean, most frequent, iterative.
            \item Encoding methods: label encoding, one-hot encoding.
            \item Skewness correction: Box-Cox, Yeo-Johnson.
            \item Scaling: RobustScaler, MinMaxScaler.
            \item Compared full feature sets and subsets selected by SFS.
        \end{itemize}
        \item These experiments allowed us to identify the best preprocessing pipeline for optimal model performance.
    \end{itemize}
\end{frame}

\section{Model Training and Evaluation}
\begin{frame}{Model Training and Evaluation}
    \begin{itemize}
        \item Trained and compared six regression models (Random Forest, AdaBoost, XGBoost, CatBoost, KNN, Decision Tree) using 10-fold cross-validation.
        \item Compared each model across all dataset and target versions (original and unskewed).
        \item Top-performing algorithms: CatBoost and RandomForest (highest mean $R^2$ scores).
        \item Sequential feature selection and preprocessing significantly improved performance compared to baseline.
    \end{itemize}
\end{frame}

\begin{frame}{Best Dataset Versions: V7 and V8}
    \begin{itemize}
        \item \textbf{V7 and V8 emerged as the best-performing dataset versions in our experiments.}
        \item \textbf{Version V7:}
        \begin{itemize}
            \item Used the optimal feature subset selected by Sequential Forward Selection (SFS).
            \item Applied mean imputation for numerical features and most frequent imputation for categorical features.
            \item Encoded categorical features with label encoding.
            \item Scaled numerical features using RobustScaler.
        \end{itemize}
        \item \textbf{Version V8:}
        \begin{itemize}
            \item Built on V7 preprocessing steps.
            \item Additionally applied Yeo-Johnson power transformation to unskew key numerical features: \texttt{kilometer}, \texttt{carAge}, \texttt{powerPS}, \texttt{adLifespan}.
        \end{itemize}
        \item Both versions enabled our top-performing models to achieve the highest accuracy and generalization in price prediction.
    \end{itemize}
\end{frame}

\begin{frame}{Hyperparameter Tuning and Final Results}
    \begin{itemize}
        \item Performed grid search for hyperparameter optimization on best pipelines (RandomForest and CatBoost).
        \item Best RandomForest: $R^2$ score = 0.835 (test), 0.976 (train) on V8 (unskewed).
        \item Best CatBoost: $R^2$ score = 0.853 (test), 0.876 (train) on V8 (unskewed).
        \item Both models showed strong generalization; no significant overfitting observed.
        \item CatBoost slightly outperformed RandomForest on the test set.
    \end{itemize}
\end{frame}

\section{Classification Problem: Predicting Price Categories}
\begin{frame}
    \centering
    \Huge
    Turning Price Prediction Into a Classification Problem
\end{frame}


\begin{frame}{From Regression to Classification}
    \begin{itemize}
        \item In addition to regression, we transformed the problem into classification by categorizing car prices.
        \item Used \texttt{pd.cut} to bin the continuous price into four categories:
        \begin{itemize}
            \item low (\textless 5,000~€)
            \item low average (5,000–10,000~€)
            \item middle average (10,000–50,000~€)
            \item high (\textgreater 50,000~€)
        \end{itemize}
        \item This allows us to analyze model performance for distinct price segments and understand price ranges better.
    \end{itemize}
\end{frame}


\begin{frame}{Data Preparation for Classification}
    \begin{itemize}
        \item Dropped the original price column, used the new price category as target.
        \item Performed an 80/20 train-test split before scaling or encoding to avoid data leakage.
        \item Oversampled rare fuel types ("hybrid" and "elektro") in the training set (20$\times$) to improve learning for underrepresented categories.
        \item Created new dataset versions (V9, V10) using:
        \begin{itemize}
            \item Feature selection (from regression task)
            \item Imputation for missing values
            \item One-hot encoding and dimensionality reduction (LDA) for categorical features
            \item Scaling and unskewing for numerical features
        \end{itemize}
    \end{itemize}
\end{frame}


\begin{frame}{Classification Models and Evaluation}
    \begin{itemize}
        \item Evaluated several classifiers with cross-validation:
        \begin{itemize}
            \item RandomForest, XGBoost, CatBoost, DecisionTree, AdaBoost, KNeighbors, Logistic Regression, GaussianNB
        \end{itemize}
        \item Used weighted F1-score as the main metric due to class imbalance.
        \item Hyperparameter tuning performed for the best pipelines (RandomForest, XGBoost) with GridSearchCV.
    \end{itemize}
\end{frame}

\begin{frame}{Best Dataset Versions: V8 and V10}
    \begin{itemize}
        \item \textbf{Versions V8 and V10 achieved the best model performance in our experiments.}
        \item \textbf{Version V8:}
        \begin{itemize}
            \item Used the optimal feature subset from Sequential Forward Selection (SFS).
            \item Applied mean imputation to numerical features and most frequent imputation to categorical features.
            \item Encoded categorical features with label (ordinal) encoding.
            \item Scaled numerical features using RobustScaler.
            \item Applied Yeo-Johnson power transformation to unskew \texttt{kilometer}, \texttt{carAge}, \texttt{powerPS}, \texttt{adLifespan}.
        \end{itemize}
        \item \textbf{Version V10:}
        \begin{itemize}
            \item Built on V9, which used the same feature selection, imputation, and robust scaling.
            \item Applied one-hot encoding to categorical features, followed by LDA for dimensionality reduction.
            \item Combined numerical features with LDA components.
            \item Also applied Yeo-Johnson power transformation to unskew key numerical features.
        \end{itemize}
    \end{itemize}
\end{frame}


\begin{frame}{Classification Results}
    \begin{itemize}
        \item \textbf{Best overall performance:}
        \begin{itemize}
            \item XGBoostClassifier on V8: F1-score (test) = 0.875
            \item RandomForestClassifier on V8: F1-score (test) = 0.874
        \end{itemize}
        \item Other strong models: CatBoost, DecisionTree.
        \item Extensive preprocessing and oversampling were critical for achieving balanced accuracy across all price categories.
        \item Both best models showed good generalization, with only mild overfitting.
    \end{itemize}
\end{frame}


\begin{frame}{Key Takeaways: Classification Task}
    \begin{itemize}
        \item Transforming price prediction to classification allowed us to focus on distinct market segments.
        \item Proper binning, feature selection, dimensionality reduction, and robust preprocessing are essential for strong performance.
        \item RandomForest and XGBoost achieved the highest F1-scores, showing strong capability for price category prediction.
    \end{itemize}
\end{frame}

\begin{frame}
    \centering
    \Huge
    Thank you! \\
    Questions?
\end{frame}

\end{document}

