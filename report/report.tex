\documentclass[conference]{IEEEtran}
\usepackage[utf8]{inputenc}
\usepackage{graphicx}
\usepackage{amsmath, amssymb}
\usepackage{caption}
\usepackage{subcaption}
\usepackage{hyperref}
\usepackage{booktabs}
\usepackage{float}      % for [H] float placement
\usepackage{placeins}
\usepackage{cite}
\hypersetup{
    colorlinks=true,
    linkcolor=blue,
    urlcolor=blue,
    }

\title{Predicting Prices for Used Cars}
\author{
    \IEEEauthorblockN{Julios Fotiou}
    \IEEEauthorblockN{Andreas Hadjoullis}
    \IEEEauthorblockA{
        Computer Science \\
        University of Cyprus \\
    }
}

\begin{document}

\maketitle

\section{Goals}
The automotive market has a wide range of car prices depending on factors such
as brand, age, mileage, and vehicle type. Predicting car prices accurately is
valuable for both buyers and sellers. This project aims to build a predictive
model using structured automotive data.

Our dataset is derived from
\href{https://www.kaggle.com/datasets/thedevastator/uncovering-factors-that-affect-used-car-prices/data}{Kaggle}
in which we have a comprehensive collection of valuable data about used cars,
and provides insight into how the cars are being sold, what price they are
being sold for, and all the details about their condition.

This project focuses on predicting car prices using machine learning models
trained on our real-world dataset. To improve prediction quality, extreme
values outside the practical range [1,000 – 200,000] are excluded. Various
regression models are evaluated using standardized preprocessing and feature
selection pipelines. We have also tested various classification algorithms on
our dataset, after splitting our dataset into bins, decided by their price.

The main goals of the project are:
\begin{itemize}
    \item Understand our dataset.
    \item Clean and preprocess raw car listing data.
    \item Select relevant features and remove noise.
    \item Train and compare multiple regression and classification models.
    \item Evaluate model performance using robust statistical metrics.
\end{itemize}

\section{Approach}
\subsection{Exploratory Data Analysis (EDA)}
First of all we need to understand our dataset. Before doing that however, we
get rid of all rows/instances in which the price does not belong in the range
of [1,000 – 200,000]. Our rezoning behind this choice, is that we only find our
model practical for values that lie in said range, and all other rows would
make it harder for the models to predict accurately prices in our desired
target range. After removing said rows, we end up with 288,023 rows instead of
the original 371,528.

We start by checking for features that have little to no deviation. For example
'offerType' only has two unique values, and one of them appears four times.
Meaning this feature has no impact on the target of our dataset. The same
applies for 'seller' and 'nrOfPictures'. So we remove the aforementioned
features after also removing the insignificantly few rows that had a different
value. We are now only working with 17 features/columns.

Some features, such as, 'vehicleType', 'gearbox', 'model', 'fuelType' and
'notRepairedDamage', have a significant number of missing values into the tens
of thousands.

\subsubsection{Car Name}

\begin{table}[H]
\centering
\resizebox{\linewidth}{!}{%
\begin{tabular}{lll}
\toprule
name & model & brand \\
\midrule
Golf\_3\_1.6 & golf & volkswagen \\
A5\_Sportback\_2.7\_Tdi & NaN & audi \\
Jeep\_Grand\_Cherokee\_"Overland" & grand & jeep \\
GOLF\_4\_1\_4\_\_3TÜRER & golf & volkswagen \\
Skoda\_Fabia\_1.4\_TDI\_PD\_Classic & fabia & skoda \\
BMW\_316i\_\_\_e36\_Limousine\_\_\_Bastlerfahrzeug\_\_Export & 3er & bmw \\
Peugeot\_206\_CC\_110\_Platinum & 2\_reihe & peugeot \\
VW\_Derby\_Bj\_80\_\_Scheunenfund & andere & volkswagen \\
Ford\_C\_\_\_Max\_Titanium\_1\_0\_L\_EcoBoost & c\_max & ford \\
VW\_Golf\_4\_5\_tuerig\_zu\_verkaufen\_mit\_Anhaengerkupplung & golf & volkswagen \\
\bottomrule
\end{tabular}
}
\caption{Sample of car name, model, and brand from dataset}
\label{tab:car_name}
\end{table}

Our first difficulty comes when dealing with the feature 'name'. As shown from
the Table~\ref{tab:car_name}, each value is distinct and does not follow any
specific convention. This is important since 'model' can be derived from 'name'
and model has 12,148 missing values. We manage to derive the model of the cars
from their name by applying fuzzy match of the name into known models of the
corresponding brand. This is important since the price of a car within the same
brand can vary largely due to the specific model, as shown here 



\section{Milestones}
Break down the key stages or checkpoints in your project timeline. For example:
\begin{itemize}
    \item Literature review
    \item Prototype implementation
    \item Testing and iteration
    \item Final deployment
\end{itemize}

\section{Experimental Setup}
Describe the experimental environment:
\begin{itemize}
    \item Hardware and software used
    \item Datasets or simulations
    \item Parameters or configurations
\end{itemize}

\section{Results and Evaluation}
Present your findings:
\begin{itemize}
    \item Quantitative results in tables/graphs
    \item Qualitative insights
    \item Comparison with baseline or existing methods
\end{itemize}

\begin{figure}[!ht]
    \centering
    % \includegraphics[width=0.9\linewidth]{example_figure.png}
    % \caption{Example caption for a figure}
    % \label{fig:example}
\end{figure}

\section{Discussion}
Interpret your results. What do they mean? Any surprising findings?

\section{Conclusion and Future Work}
Summarize your project outcomes and propose what could be improved or continued in future research.

\end{document}
